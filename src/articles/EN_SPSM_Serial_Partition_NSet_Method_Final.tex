%===============================SPSM - Serial Partition od an n-Set Method=============================
\documentclass {amsart}
\usepackage[latin1]{inputenc} %permite o uso de acentos
\newcommand{\emerson}{Emerson A. de O. Lima}
\newcommand{\glaucio}{Glaucio G. de M. Melo}

%=================================Preamble ends here========================================
\begin{document}
\title[Serial Partition of an {$n$}-Set Method]
 {Serial Partition of an {$n$}-Set Method}

%=====================================Title=================================================
\author[Melo]{\glaucio}
\address[Melo]{Departamento de Estat\'{\i}stica e Inform\'{a}tica - UNICAP}
\email[Melo]{glaucio@dei.unicap.br}
%=====================================Melo==================================================

\author[Oliveira-Lima]{\emerson}
\address[Oliveira-Lima]{Departamento de Estat\'{\i}stica e Inform\'{a}tica - UNICAP}
\email[Oliveira-Lima]{eal@dei.unicap.br}
%================================Oliveira Lima==============================================

\keywords{Combinatorial Algorithms, Complexity, Optimization,
Partition}
%===================================Key Words===============================================

\begin{abstract}
This article presents the \emph{Serial Partition of an n-Set
Method} (SPSM). The method's proposal is to obtain the partition
of an {$n$}-Set from its serial number. The method that does the
inverted SPSM process will be also shown, getting the serial
number from the partition vector. The article concludes with a
demonstration of the combinatorial properties from the material
exposed here.
\end{abstract}
%=================================End Abstract=============================================
\maketitle
%=================================Introduction=============================================
\section*{Introduction}

For the sets partitions, we consider a family of subsets {$T_1,
T_2,\ldots,T_k $} contained in a set {$S = \{1,2,\ldots,n\}$} that
satisfy the conditions:

\begin{equation}
T_i \cap T_j = \oslash \indent (i \neq j)
\end{equation}
\begin{equation}
\bigcup _{i=1}^{k} T_i = S
\end{equation}
\begin{equation}
T_i = \oslash \indent (i = 1, 2, \ldots,k)
\end{equation}

We don't consider the order of the elements contained on each one
of {$S$} subsets. The algorithm \emph{Next Partition of an
n-Set}\cite{wi} finds the next partition set from the current one,
working only with local information. The article's proposal is to
obtain a partition of {$S$} related to its position on the list of
partition, without considering the information about the partition
vector.

%==================== SPSM Construction =====================================
\section*{SPSM Construction}

The SPSM was built from the analysis of the data output of the
\emph{Next Partition of an {$n$}-Set} algorithm. Each index that
denotes the owned elements to a subset has a pattern model and the
search for these indexes is made by a combinatorial structure that
take as basis a tree that represents all the partitions of a set
with {$n$} elements, called Bell Tree\cite{me}. The number of
nodes of this structure grows quickly when {$n$} increases. So,
the solution for such problem is to define a new structure that
does a mapping of the tree specific positions. As a reference we
took a matrix structure to keep those positions, called Matrix
{$D$}.


%==================== Definition of Matrix =====================================
\subsection*{Definition of Matrix {$D$}}

The Matrix {$D$} is a superior triangular matrix with {$n$} x
{$n$} dimension. Its first column is made by the Bell
Numbers\cite{me}

\begin{equation}
    D_{v,0} = B_{n-v}
\end{equation}

and the others columns are defined by the equation below:

\begin{equation}
    D_{u,v} = D_{u,v-1} - v.D_{u+1,v-1}
\end{equation}

Where {$u$} indicates the line and {$v$} the column on {$D$}. The
matrix showed on \ref{md} represents the Matrix {$D$} for {$n=6$}.

\begin{equation} \label{md}
D = \left(\begin{array}{c c c c c c}
203 & 151 & 77 & 26 & 6 & 1 \\
52  & 37  & 17 & 5  & 1 &     \\
15  & 10  & 4  & 1  &   &     \\
5   & 3   & 1  &    &     &   \\
2   & 1   &      &      &     &   \\
1   &         &      &      &       &   \\
\end{array}\right)
\end{equation}


%==================== Algorithm =====================================
\subsection*{Algorithm for Matrix {$D$} fulfilling}

Before process the SPSM, it is necessary fill in the components of
Matrix {$D$} to speed up the attainment calculus of the search on
the Bell Tree. The use of Matrix {$D$} abstracts the construction
of the whole Bell Tree, processing only the necessary to the SPSM
use. If the tree had been built completely, we would have the
number of nodes {$R$} equivalent to

\begin{equation} \label{eqb}
R = \sum_{j=1}^{n}B_j
\end{equation}

whereas on the Matrix {$D$}, we have

\begin{equation} \label{eqb}
R = \frac{n^2 + n}{2}
\end{equation}
\\\\
\textbf{Algorithm for Matrix {$D$} fulfilling}\\
Algorithm Specifications:

\begin{itemize}
\item {$D$}: Matrix that keep the specific positions of the Bell
Tree;

\item {$i,j$}: Indexes for the raster of Matrix {$D$};

\item{$B_n$}: Bell Number.

\end{itemize}
Routine: \\
\indent For {$i \leftarrow 0$} to {$n-1$} do\\
\indent \indent {$D_i,0 \leftarrow B_{n-i}$}\\
\indent End For\\
\indent For {$i \leftarrow 1$} to {$n-1$} do\\
\indent \indent For {$j \leftarrow 0$} to {$n-i-1$}\\
\indent \indent \indent {$D_{j,i} \leftarrow D_{j,i-1} - i.D_{j+1,i-1}$}\\
\indent \indent End For\\
\indent End For.


%==================== SPSM Spec. =====================================
\subsection*{SPSM's Specifications}

After that the Matrix {$D$} has its values filled in, the SPSM can
be invoked. We have an external loop that runs all the partition
vector, attributing for each component by means of the sub-routine
\emph{element} that does a raster on the current tree level (the
structure was abstracted from the method). On this raster, we have
a condition that examines if the variable used for the indication
of extrapolation surpassed the serial number token as an input. If
it exceed this value, we go down a tree level; if not, we add the
descendants' current value to the control variable for later
verification of extrapolation, token as an input. If the condition
had not been satisfied for the whole loop, it means that the
search on the tree's level arrived to the last descendant of this
one, indicating that the search will be expanded to the last
descendant of the current tree's level.
\\\\
\textbf{Serial Partition of an {$n$}-Set Method}\\
Algorithm Specifications:

\begin{itemize}

\item {$p$}: Partition vector;

\item {$D$}: Matrix that keep the specific positions of the tree;

\item {$i,j$}: Indexes for the raster of Matrix {$D$};

\item {$k$}: Index for the raster on the vector {$p$};

\item {$t$}: Index that determines the element of each component
of partition vector;

\item {$a$}: Number that does the convergence for the serial
number given as input;

\item {$s$}: Partition's serial.

\end{itemize}
Routine: \\
\indent {$i \leftarrow j \leftarrow a \leftarrow 0$}\\
\indent For {$k \leftarrow 0$} to {$n-1$} do\\
\indent \indent {$p_k \leftarrow element$}\\
\indent End For \\
return p.\\\\
Sub-Routine {$element$}\\
\indent For {$t \leftarrow 0$} to {$j$} do\\
\indent \indent If {$a + D_{i,j} \geq s$}\\
\indent \indent \indent {$i \leftarrow i + 1$}\\
\indent \indent \indent return t.\\
\indent \indent End If\\
\indent \indent Else\\
\indent \indent \indent {$a \leftarrow a + D_{i,j}$}\\
\indent \indent End Else \\
\indent End For\\
\indent {$j \leftarrow j + 1$} \\
return {$j$}.

%==================== Stylized output here =====================================
\subsection*{Algorithm for stylized output of the partition vectors}

We know that the output gotten on both SPSM and Next Partition of
an {$n$}-Set corresponds to the indexes of each subset of the
partition on the set. For the partition vector on position {$26$},
we have the output {$(0,1,1,0,0)$} and the stylized output
corresponds to {$(1,4,5)(2,3)$}, indicating that {$1,4$} and {$5$}
is on the first subset (indicated as {$0$}), with {$2$} and {$3$}
on the second subset (indicated as {$1$} on the partition vector).
For show the stylized output, it was used a \emph{string} vector
to get the elements adequately on each subset which it belongs.
Next, the algorithm does the output treatment and the all elements
of each subset be ordered related to the last elements of
partition as an input for the stylized output of the {$n$}-sets'
partitions method.
\\\\
\textbf{Algorithm for stylized output data}\\
Algorithm Specifications:

\begin{itemize}
\item {$s$}: {$Strings$} vector that does the mapping of the
elements referring to the partition vector's indexes;

\item {$p$}: Partition vector;

\item {$i$}: Indexes for the raster of vector {$s$};

\item {$r$}: Final output on {$string$} form;

\item {$n$}: Size of the partition vector;

\item {$+$}: Operator that denotes a concatenation between
{$strings$};

\item {$k$}: {$String$} that get the current element of the
stylized output;

\item {$length(w)$}: Function that returns the size of the
{$string$} {$w$};

\item {$substring(w,ini,sup)$}: Function that returns a {$w$}
{$substring$} of the interval between {$ini$} and {$sup$}.

\end{itemize}
Routine:\\
\indent For {$i \leftarrow 0$} to {$n-1$} do\\
\indent \indent {$s_{p_i} \leftarrow s_{p_i} + (i + 1) + ","$}\\
\indent End For\\
\indent For {$i \leftarrow 0$} to {$n-1$} do\\
\indent \indent If {$length(s_i) > 1$}\\
\indent \indent \indent {$k \leftarrow substring(s_i$}, {$0$}, {$length(s_i) - 1)$}\\
\indent \indent End If\\
\indent \indent {$s_i \leftarrow "(" + k + ")"$}\\
\indent \indent If {$s_i = "()"$}\\
\indent \indent \indent {$r \leftarrow r + s_i$}\\
\indent \indent End If\\
\indent End For\\
return r.

%==================== The Inverse Process =====================================
\section*{Inverted Process of SPSM}

For the inverted process of SPSM, we have a partition vector as
input and the serial number as output. Such process is made taking
each partition vector's component as the number of loops that can
does a sum of each specific position referring to the current node
of the tree mapping through the Matrix {$D$}.

%==================== Specifications of Inverse Proc.. =====================================
\subsection*{Inverted Process of SPSM Specifications}

Initially, we have the indexes attribution referring to the Matrix
{$D$}. The index {$u$} has the attribution equals to {$1$} because
it is not necessary do the mapping on the first component of the
Matrix {$D$}, once it always begins with zero for being the main
tree descendant. After that, we have an external loop that runs
the whole partition vector given as input, with another loop that
does the sum to the main positions' serial mapped on Matrix {$D$}.
After the sum attributions to the serial, we have the checking if
the actual component should go down a line on Matrix {$D$}
(equivalent to go down a level on the tree). If it should go down
a level, the \textbf{line} is added by one unit. If not, the
\textbf{column} on {$D$} is added by one unit, indicating the
search to the descendants on the current node reached its last one
and the search be doing on the other tree ramification.
\\\\
\textbf{Inverted Process of SPSM}\\
Algorithm Specifications:
\begin{itemize}
\item {$p$}: Partition vector, given as input;

\item {$i,j$}: Indexes referring to the raster on vector {$p$};

\item {$D$}: Matrix of specific positions on the Bell Tree;

\item {$u,v$}: Indexes referring to the mapping on {$D$};

\item {$s$}: Partition's serial, that is the result expected from
de algorithm.
\end{itemize}
Routine: \\
\indent {$i \leftarrow j \leftarrow v \leftarrow 0$}\\
\indent {$s \leftarrow u \leftarrow 1$}\\
\indent For {$i \leftarrow 1$} to {$n-1$} do\\
\indent \indent For {$j \leftarrow 0$} to {$p_i - 1$} do\\
\indent \indent \indent {$s \leftarrow s + D_{u,v}$}\\
\indent \indent End For\\
\indent \indent If {$j \leq p_{i-1}$}\\
\indent \indent \indent {$u \leftarrow u + 1$}\\
\indent \indent End if\\
\indent \indent Else\\
\indent \indent \indent {$v \leftarrow v + 1$}\\
\indent \indent End Else \\
\indent End For \\
return s.

\section*{Demonstration}
(This section is not ready)

%===========================References==============================================
\begin{thebibliography}{99}
\bibitem{wi} WILF, Herbert S., NIJENHUIS, A. ~Combinatorial Algorithms for computers and calculators. Academic
Press, INC, 1978.
\bibitem{me} MELO, Glaucio G. de M., OLIVEIRA-LIMA, Emerson A. de O. Serial Partition of an n-Set Method,
(Not published yet),2004.
\end{thebibliography}
\end{document}
