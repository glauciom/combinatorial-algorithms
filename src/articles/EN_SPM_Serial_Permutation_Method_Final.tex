%===============================Serial Permutation Method (SPM)=============================
\documentclass {amsart}
\usepackage{color}
\usepackage[latin1]{inputenc}
\newcommand{\emerson}{Emerson A. de O. Lima}
\newcommand{\glaucio}{Glaucio G. de M. Melo}
\newcommand{\azero}{\textcolor{blue}{0}}
\newcommand{\aum}{\textcolor{blue}{1}}
\newcommand{\adois}{\textcolor{blue}{2}}
\newcommand{\rzero}{\textcolor{red}{0}}
\newcommand{\rum}{\textcolor{red}{1}}
\newcommand{\rdois}{\textcolor{red}{2}}
%=================================Preamble ends here========================================
\begin{document}
\title[Serial Permutation Method]
 {Serial Permutation Method}
%=====================================Title=================================================
\author[Melo]{\glaucio}
\address[Melo]{Departamento de Estat\'{\i}stica e Inform\'{a}tica - UNICAP}
\email[Melo]{glaucio@dei.unicap.br}
%=====================================Melo==================================================
\author[Oliveira-Lima]{\emerson}
\address[Oliveira-Lima]{Departamento de Estat\'{\i}stica e Inform\'{a}tica - UNICAP}
\email[Oliveira-Lima]{eal@dei.unicap.br}
%================================Oliveira Lima==============================================
\keywords{Combinatorial Algorithms, Complexity, Combinatorial
Optimization, Permutation}
%===================================Key Words===============================================
\begin{abstract}
This article presents the \emph{Serial Permutation Method} (SPM).
The SPM proposal is to obtain a permutation vector from its serial
number. The algorithm that does the SPM's inverted method is also
showed, getting the serial number from the permutation vector. %The
%article ends with the demonstration of the method offered here.
\end{abstract}
%=================================End Abstract=============================================
 \maketitle
%=================================Introduction=============================================
\section*{Introduction}
The permutation algorithms are classified in two groups: the one
that creates a set of permutations from the identity permutation
and the other that produces a set of permutations by means of
simple changes between the vector's elements, creating a new
permutation from the previous one. The algorithm \emph{Next
Permutation for N letters} \cite{wi} belongs to the second group,
creating a complete set of permutations through successive
algorithm invocation, getting the next permutation until reaches
the last vector of the list. The algorithm \emph{Next Permutation}
is powerful creating the next permutation vector using only local
information. This guarantees that the next vector will be
different from anyone else that was defined before.

There is a specific problem on this creating process:
\begin{quote}
\emph{Is it possible to get directly a permutation vector located
on a specific position on the list of vectors, excluding the
alternative of the Next Permutation algorithm's successive
execution until gets the desired vector?}
\end{quote}

The \emph{Serial Permutation Method} (SPM) answers this question,
being able to process the same list of permutation that the
\emph{Next Permutation} algorithm does, with a difference: the SPM
only needs the serial number and the permutation vector's size to
process the desired output. It is also possible to invert this
process from the SPM, getting the serial number through the
permutation vector.
%===========================Construction====================================================
\section*{SPM Construction}
%===========================================================================================
The SPM was developed from the observation of the \emph{Next
Permutation's} data outputs. This algorithm has an auxiliary
variable called offset vector, which is defined below \cite{wi}.

Consider {$p$} the permutation vector with {$n$} elements and let
{$d$} be the offset vector with {$n - 1$} elements, the vector
{$d$} is defined by:
\begin{equation} \label{eq}
    d_i = |j: j \leq i, ~p_j > p_{i+1}| ~~i = 1, 2, ... ,n
\end{equation}
That is explained:
\begin{quote}
\emph{How many numbers are there biggest than {$p_{i+1}$} between
the beginning of the permutation vector and the {$i$} index of the
{$d$} vector?}
\end{quote}

Being {$n!$} the complete list of the created vectors, we have in
Table \ref{tab_1} an example of output from the {$p$} permutation
vector related to its serial, for {$n = 5$} and the {$serial =
32$}.
\begin{table}[!hbp]
  \centering
  \caption{Input and output processing that are common in the \emph{Next Permutation} algorithm.}
  \label{tab_1}
\begin{tabular}{c|c}
 \hline
  Input & Output \\
  \hline
  {$n = 5$} & {$p = (3, 5, 1, 2, 4)$} \\
  {$Serial = 32$} & {$d = (0, 2, 2, 1)$} \\
  \hline
 \end{tabular}
\end{table}

The SPM is subdivided in two steps, and the data input of the
second step corresponds to the data output of the first step:
\begin{enumerate}
\item  Given a serial number and the size of the permutation
vector, determine the offset vector;
\item Given the offset vector,
determine the permutation vector.
\end{enumerate}
%=========================Serial Offset Algorithm===========================================
\subsection*{Algorithm to Attain the Offset Vector}
%===========================================================================================
To solve the first SPM step it is necessary an algorithm to get
the offset vector, which is called \emph{Serial Offset Algorithm}
(SOA). The algorithm does a calculus that reflects the pattern of
the offset vectors' creation throughout the whole list of
permutation, using only local information. In Table \ref{table},
we can visualize the pattern of the offset vector in the whole
list of permutation, where the detached column will serve as a
guide to the determination that the SOA decodes.
\begin{table}
  \centering
  \caption{Full list of permutations, with {$n = 4$}}
  \label{table}
\begin{tabular}{c|c|c}
 \hline
   Serial  &p  &d\\
 \hline
    1   &(1,2,3,4)  &(0,\azero,0)\\
  \hline
    2   &(2,1,3,4)  &(1,\azero,0)\\
  \hline
    3   &(3,1,2,4)  &(1,\aum,0)\\
  \hline
    4   &(1,3,2,4)  &(0,\aum,0)\\
  \hline
    5   &(2,3,1,4)  &(0,\adois,0)\\
  \hline
    6   &(3,2,1,4)  &(1,\adois,0)\\
  \hline
    7   &(4,2,1,3)  &(1,\rdois,1)\\
  \hline
    8   &(2,4,1,3)  &(0,\rdois,1)\\
  \hline
    9   &(1,4,2,3)  &(0,\rum,1)\\
  \hline
    10  &(4,1,2,3)  &(1,\rum,1)\\
  \hline
    11  &(2,1,4,3)  &(1,\rzero,1)\\
  \hline
    12  &(1,2,4,3)  &(0,\rzero,1)\\
  \hline
    13  &(1,3,4,2)  &(0,\azero,2)\\
  \hline
    14  &(3,1,4,2)  &(1,\azero,2)\\
  \hline
    15  &(4,1,3,2)  &(1,\aum,2)\\
  \hline
    16  &(1,4,3,2)  &(0,\aum,2)\\
  \hline
    17  &(3,4,1,2)  &(0,\adois,2)\\
  \hline
    18  &(4,3,1,2)  &(1,\adois,2)\\
  \hline
    19  &(4,3,2,1)  &(1,\rdois,3)\\
  \hline
    20  &(3,4,2,1)  &(0,\rdois,3)\\
  \hline
    21  &(2,4,3,1)  &(0,\rum,3)\\
  \hline
    22  &(4,2,3,1)  &(1,\rum,3)\\
  \hline
    23  &(3,2,4,1)  &(1,\rzero,3)\\
  \hline
    24  &(2,3,4,1)  &(0,\rzero,3)\\
  \hline
 \end{tabular}
\end{table}
Let {$d$} be the offset vector with {$n$} elements. Each vector's
{$d$} column showed in Table \ref{table} has the following
properties:
\begin{itemize}
\item Let {$k$} be the value of the current column ({$k = 1, 2,
..., n$}), the elements of the {$d_k$} column ({$0 \leq d_k \leq
k$}) repeat {$k!$} times until the element of the column gets the
{$k$} value. This can be visualized on the first part of the
detached column in Table \ref{table}, illustrated in blue, which
the number of detached elements in blue or red is defined by {$(k
+ 1)!$} \item When {$d_k = k$} in {$k!$} times, the following
elements from the list are put on the inverted form, like showed
in red in Table \ref{table}. The direct and inverted list
intercalate themselves until complete all the positions of the
column.
\end{itemize}

Exists throughout each offset vector column an intercalation
between the two kinds of lists (direct and inverted), we can
consider this fact as an element of parity in the list,
considering as an even the list in its direct form and as odd the
list in its inverted form. To determinate the relation between the
serial number and the parity of the list, follows:
\begin{equation}
f =
\displaystyle\biggl\lfloor\frac{s-1}{(k+1)!}\displaystyle\biggl\rfloor
~mod~2
\end{equation}

Where:
\begin{itemize}
    \item {$f$}: Determines if the list is direct {$(f = 0)$} or inverted {$(f = 1)$};
    \item {$s$}: Serial number;
    \item {$k$}: Index of the offset vector;
    \item {$\lfloor x \rfloor$}: Floor function. It returns the biggest integer value smaller than {$x$};
    \item {$mod$}: An operation that returns a division's rest.
\end{itemize}

If the list had been direct, we attribute to the elements of the
offset vector:

\begin{equation}    \label{eqt}
d_k = \displaystyle\biggl\lfloor\frac{[(s-1) ~mod~
(k+1)!]}{k!}\displaystyle\biggl\rfloor
\end{equation}

If the list had been inverted, we attribute the complement which
would be the direct list:

\begin{equation}    \label{eqq}
d_k = k - \displaystyle\biggl\lfloor\frac{[(s-1) ~mod~
(k+1)!]}{k!}\displaystyle\biggl\rfloor
\end{equation}

Each attribution is made through a loop that go through the offset
vector.
\newline\newline
\textbf{Serial Offset Algorithm (SOA)}
\newline
Routine Specifications:
\begin{itemize}
    \item {$n$}: Size of the offset vector;
    \item {$i$}: Index of the offset vector;
    \item {$d$}: Offset vector, alternating its indexes from {$0..n-1$};
    \item {$s$}: Permutation's serial.
\end{itemize}
Routine:
\newline
  \indent For {$i \leftarrow 1$} to {$n$} do \newline
  \indent\indent If {$ \lfloor (s - 1) ~/~ (i+1)! \rfloor ~mod~ 2 = 1$} \newline
  \indent\indent\indent {$d_{i-1} \leftarrow i - \lfloor((s-1) ~mod~ (i+1)!) ~/~i!)\rfloor$} \newline
  \indent\indent Else \newline
   \indent \indent\indent {$d_{i-1} \leftarrow \lfloor((s-1) ~mod~ (i+1)!)
   ~/~ i!)\rfloor$} \newline
   \indent End For \newline \indent
   return {$d$}.
%================================Permutation Algorithm by Offset===========================
\subsection*{Algorithm to attain the permutation vector from the offset vector}
%==========================================================================================
After the SOA is computed, the SPM is concluded with the
\emph{Permutation Algorithm by Offset} (PAO). The PAO does the
SPM's second step, returning the desired output. A relevant topic
for the PAO construction is to find the decoding process of the
permutation vector, with only the offset vector being the input.
We know that the offset vector maps the elements of the
permutation vector. In the decoding, we have:

Let {$p$} be the permutation vector and {$d$} the offset vector:
\begin{equation}
    p = (p_1, p_2, ..., p_n)  \indent d = (d_1, d_2, ..., d_{n-1})
\end{equation}

And {$d$} already has its values determining by the SOA. This is
how the decoding process is made: we know that {$d_1$} has its
values discretely included between {$0$} and {$1$}. Obviously,
{$0$} and {$1$} are the unique possible elements for {$d_1$}. From
this information, we can conclude:

\begin{displaymath}
d_1 = \left\{ \begin{array}{ll}
0, & \textrm{if $p_1 < p_2$}\\
1, & \textrm{if $p_2 < p_1$}\\
\end{array} \right.
\end{displaymath}
\newline

We do not consider the possibility of equality between the
elements of the permutation vector, because we know that there is
no repeated elements on the vector, and if we organize it, the
difference between them will be only one unit. Like {$d_1$}, the
element {$d_2$} has its values included between {$0$} and {$2$}.
In this and in other cases, we analyze the current inequality from
the inequality that was created previously. For {$d_2$}, we have:

\begin{displaymath}
d_1 = \left\{ \begin{array}{ll}
0, & d_2 = \left\{
\begin{array}{ll}
0, & \textrm{if $p_1 < p_2 < p_3$}\\
1, & \textrm{if $p_1 < p_3 < p_2$}\\
2, & \textrm{if $p_3 < p_1 < p_2$} \\
\end{array} \right.
\\\\
1, & d_2 = \left\{
\begin{array}{ll}
0, & \textrm{if $p_2 < p_1 < p_3$}\\
1, & \textrm{if $p_2 < p_3 < p_1$}\\
2, & \textrm{if $p_3 < p_2 < p_1$} \\
\end{array} \right.\\
\end{array} \right.
\end{displaymath}
\newline

We can conclude that as the value of {$d_2$} increases, {$p_3$}
"slides" on the left through the inequality. In general, we have:
\begin{equation}
d_k \in \{0, 1, ..., k\} \indent \{p_i < p_j < ... < p_k < ... <
p_t\} \indent i, j, t \neq k
\end{equation}

With {$p_k$} in the inequality with {$d_k$} positions, counted
from right to left, because the order of the elements is
ascendent. Finished the offset vector's raster, we will get a set
of inequalities that informs the order of the permutation
elements. We have, for example:

Given a offset vector {$d = (0,2,2)$}, the inequality for a {$p$}
permutation vector is:
\begin{equation} \label{eq:des}
p_3 < p_4 < p_1 < p_2
\end{equation}

Being the last inequality at \ref{eq:des} the final disposition
between the elements of the vector. After the mapping of the
permutation's elements was done, we can say that each element of
the vector ordered in \ref{eq:des} corresponds to the elements of
the identity permutation. This fact classifies the SPM in the
first group described in the introduction of this article. On the
\ref{eq:des} inequality, we have:
\begin{equation}
(p_3 = 1) < (p_4 = 2) < (p_1 = 3) < (p_2 = 4)
\end{equation}

The next step to get the final output is to arrange each one of
the elements ordered on its own positions. As the {$p$} vector is
ordered like {$(p_1, p_2, ..., p_n)$}, we have:
\begin{equation}
 p = (p_1, p_2, p_3, p_4)  \indent \Rightarrow \indent p = (3, 4, 1, 2)
\end{equation}

So we can get the SPM's final output. In implementation terms, to
compute the input built from the permutation vector (from right to
left) the corresponding indexes of the vector were used in
relation to its complement. We will notice at the implementation
that in the insertion moment of the current element, if it
replaces another, the elements of inequality located on the left
will "slide" to the left side, allocating space for the current
element. The "slides to the left" operation is implemented on the
subroutine \emph{push}.
\newline \newline
\textbf{Permutation Algorithm by Offset (PAO)}
\newline
Routine Specifications:
\begin{itemize}
    \item {$n$}: Size of the permutation vector;
    \item {$i,j$}: Indexes of the algorithm's vectors;
    \item {$p$}: Permutation vector, alternating its elements on {$0..n-1$};
    \item {$r$}: Vector which will keep the element's position before being ordered.
\end{itemize}
Routine:
\newline
\indent {$r_{n-1} \leftarrow 1$}
\newline
 \indent For {$i \leftarrow 1$} to {$n-1$} do \newline
  \indent \indent If {$r_{n-1-d_{i-1}} = 0$} \newline
   \indent \indent \indent {$r_{n-1-d_{i-1}}$} {$\leftarrow i+1$} \newline
  \indent \indent Else \newline
   \indent \indent \indent {$r_{n-1-d_{i-1}}$} {$\leftarrow push$} \newline
\indent End For \newline \indent For {$i \leftarrow 0$} to {$n-1$}
do \newline
   \indent \indent {$p_{r_i-1} \leftarrow i+1$} \newline
\indent  End For \newline return {$p$}.
\\\\
\textbf{Subroutine Push} \\
\indent For {$j \leftarrow (n-1) - i$}
to {$j < (n - 1) - d_{i-1}$} do
\newline
\indent \indent {$p_j \leftarrow p_{j+1}$}
\newline
\indent End For
\newline
return {$i+1$}.
%===================================Inverted Process=========================================
\section*{SPM Inverted Process}
%============================================================================================
We can describe now the inverted process of the SPM's. We have the
permutation vector as the input, and the desired output is the
correspondent serial number. The ingenuous process to get the
serial number for the permutation vector is the raster of the
permutation list, comparing the vectors one by one, until gets the
equivalent vector computed on the input, being the returned value
the loop's index that does this raster. However, we can find the
serial value inverting the SPM's steps:
\begin{enumerate}
    \item Given the permutation vector, find the offset vector;
    \item Given the offset vector, find the serial number.
\end{enumerate}

Like the SPM, the second process depends on the first, with the
first step data output corresponding to the second step data
input.
%===================================Offset Algorithm by Permutation==========================
\subsection*{Algorithm to attain the offset vector from the permutation vector}
%============================================================================================
We will call this algorithm as the \emph{Offset Algorithm by
Permutation} (OAP). It uses the definition of the offset vector
(see equation \ref{eq}). Two nested loops add the value of each
element of the offset vector.
\newline \newline
\textbf{Offset Algorithm by
Permutation (OAP)}
\newline
Routine Specifications:
\begin{itemize}
    \item {$n$}: Size of the permutation vector;
    \item {$i, j$}: Indexes of algorithms's vectors;
    \item {$p$}: Permutation vector alternating its elements on {$0..n-1$};
    \item {$d$}: Offset vector alternating its elements on {$0..n-2$}.
\end{itemize}
Routine:
\newline
\indent For {$i \leftarrow 0$} to {$n-2$} do \newline \indent
\indent For {$j \leftarrow 0$} to {$i$} do \newline \indent
\indent \indent If {$p_j > p_{i+1}$} \newline \indent \indent
\indent \indent {$d_i \leftarrow d_i + 1$} \newline \indent
\indent \indent End If \newline \indent \indent End For \newline
\indent End For \newline return {$d$}.

%=====================================Serial Algorithm by Offset=============================
\subsection*{Algorithm to attain the serial number from the offset vector}
%============================================================================================
Like the OAP, the \emph{Serial Algorithm by Offset} (SAO) also
uses definitions already showed at this article. We know that the
value attributed for the elements of the offset vector, defined on
\ref{eqt} and \ref{eqq} can be ordered depending on the serial.
Given the variables:
\begin{itemize}
    \item {$d = (d_1,d_2,...,d_k,..., d_n)$}: Offset vector;
    \item {$d_k$}: Offset vector's element;
    \item {$s$}: Permutation's serial number;
    \item {$k$}: Index of the offset vector;
    \item {$q$}: Quotient of the division between {$s-1$} and {$(k+1)!$};
    \item {$\lfloor x \rfloor$}: It returns the biggest integer value smaller than {$x$}.
\end{itemize}
For direct list, we have:
\begin{equation} \label{eqdez}
s-1 = d_k . k! + \lfloor q . (k+1)! \rfloor
\end{equation}
For inverted lists, we have:
\begin{equation} \label{eqonze}
s-1 = (k - d_k). k! + \lfloor q . (k+1)! \rfloor
\end{equation}

An important question for the algorithm implementation is how to
find the variable's {$q$} value. To solve this question we have to
consider that {$d_n$} belongs to a column that has only one list
on a direct disposition (see Table \ref{table}, last vector {$d$}
column). With this information, we conclude that the quotient for
this column corresponds to a value between zero (closed) and one
(open), making:
\begin{equation}
\lfloor q . (k+1)! \rfloor
\end{equation}

Corresponds to zero. With this,  \ref{eqdez} and \ref{eqonze} is
equivalent to:
\begin{equation}  \label{eqdoze}
 s-1 = d_k . k!
\end{equation}

For direct lists, and
\begin{equation}   \label{eqtreze}
s-1 = (k - d_k). k!
\end{equation}

For inverted lists.

The strategy of implementation to get the serial is done by
incremental mode, working the current information being based on
past information, doing a raster on the offset vector from right
to left. This is particularly useful on this case, considering the
next value that will converge to the final serial that belongs to
a group related with the previous serial elements. Each quotient
{$q$} during the raster will corresponds to zero, because we are
considering the division in relation to the gotten serial. Another
relevant question is how to classify if the previous element of
the offset vector it is contained in a direct or inverted list.
Looking at Table \ref{table}, we can make easily an equivalence
grade, showed in Table \ref{tabletres}.
\begin{table}[!hbp]
  \centering
  \caption{Next element status list of the offset vector}
  \label{tabletres}
\begin{tabular}{c|c l c}
\hline
\multicolumn{2}{c|}{Previous Value - Parity} && Next Value - Parity\\
\hline
Element & List && List\\
\hline
  Even & Direct & {$\Longrightarrow$} & Direct\\
  Even & Inverted & {$\Longrightarrow$} & Inverted\\
  Odd & Direct & {$\Longrightarrow$} & Inverted\\
  Odd & Inverted & {$\Longrightarrow$} & Direct\\
\hline
\end{tabular}
\end{table}

Exemplifying the attainment serial process from the offset vector,
we have:

Let {$d = (1,0,3)$} be the offset vector and {$s$} the permutation
serial. As the raster is done from right to left, we work first
with the value {$3$}. As was already said, we have as an initial
input a direct list which {$3$} is contained. For direct lists, we
use an equation showed on \ref{eqdoze}:
\begin{equation}
s \leftarrow 3 . 3!  \Rightarrow 18
\end{equation}

After that, we observer the next element of the offset vector. As
the previous element is in a direct list and it is odd, the
element {$0$} will be in a inverted list. Using \ref{eqtreze}, we
have:
\begin{equation}
s \leftarrow s + (2 - 0) . 2! \Rightarrow 18 + 4 \Rightarrow 22
\end{equation}

Finally, observing the last element of the offset vector, we have
the previous element which is in a inverted list and it is even.
Looking at Table \ref{tabletres}, we evidence that the element
{$1$} is in a inverted list. Thus, we have:
\begin{equation}
s \leftarrow s + (1 - 1) . 1! \Rightarrow 22 + 0 \Rightarrow 22.
\end{equation}

After we have calculated the serial in a incremental mode, we add
{$1$} to the serial, because the equations \ref{eqdoze} and
\ref{eqtreze} depend on {$s-1$}. Thus, the serial value
characterizes it between {$(1, n!)$}. Where {$n$} is the size of
the permutation value given as an input:
\begin{equation}
s \leftarrow s + 1 \Rightarrow 22 + 1 \Rightarrow 23.
\end{equation}

The serial {$23$} is the corresponding serial to the offset vector
{$(1,0,3)$} which corresponds to the permutation vector
{$(3,2,4,1)$}. On the algorithm's implementation showed here, the
increment of one unit to the serial was made in the beginning,
before get into the loop. A boolean variable was specified to
determine if the list which belongs the elements is direct or
inverted. The conditions that determine if the list is direct or
inverted is optimized from four to two conditions. The algorithm
returns a non-negative integer value, corresponding to the serial
number required.
\newline \newline
\textbf{Serial Algorithm by Offset (SAO)}
\newline
Routine Specifications:
\begin{itemize}
    \item {$s$}: Serial value;
    \item {$n$}: Size of the offset vector;
    \item {$i$}: Index of the offset vector;
    \item {$d$}: Offset vector, alternating its elements on {$0..n-1$};
    \item {$direct$}: Boolean variable. It determines if the list is direct or not.
\end{itemize}
Routine:
\newline
\indent {$s \leftarrow [~d_{n -1} . (n-1)!~] + 1$} \newline
\indent {$direct \leftarrow true$} \newline \indent
 For {$i \leftarrow n-1$} to {$1$} do \newline \indent \indent
  If {$(d_i= even ~and~ direct = true)$} {$or$} {$(d_i = odd ~and~ direct = false)$} \newline
  \indent \indent \indent
   {$direct \leftarrow true$} \newline \indent
   \indent\indent
   {$s \leftarrow s +  d_{i-1} . (i-1)!$} \newline \indent
   \indent
  End If \newline \indent \indent
  Else \newline \indent \indent
  If {$(d_i = odd ~and~ direct = true)$} {$or$} {$(d_i = even ~and~ direct = false)$}
  \newline \indent \indent \indent
   {$direct \leftarrow false$} \newline \indent \indent \indent
   {$s \leftarrow s +  (i - d_{i-1}) . (i-1)!$} \newline \indent \indent
  End If \newline \indent
 End For \newline
return s.

\section*{Demonstration}
(This section is not ready)


\begin{thebibliography}{99}
\bibitem{wi} WILF, Herbert S., NIJENHUIS, A.
~Combinatorial Algorithms for computers and calculators. Academic
Press, INC, 1978.
\end{thebibliography}
\end{document}
