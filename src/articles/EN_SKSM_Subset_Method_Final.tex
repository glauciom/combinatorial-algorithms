%===============================Serial KNSubset Method=====================================
\documentclass {amsart}
\usepackage[latin1]{inputenc}
\newcommand{\emerson}{Emerson A. de O. Lima}
\newcommand{\glaucio}{Glaucio G. de M. Melo}
%=================================Preamble ends here========================================


\begin{document}
\title[Serial {$k$}-Subset of an {$n$}-Set Method]
 {Serial {$k$}-Subset of an {$n$}-Set Method}
%=====================================Title=================================================


\author[Melo]{\glaucio}
\address[Melo]{Departamento de Estat\'{\i}stica e Inform\'{a}tica - UNICAP}
\email[Melo]{glaucio@dei.unicap.br}
%=====================================Melo==================================================


\author[Oliveira-Lima]{\emerson}
\address[Oliveira-Lima]{Departamento de Estat\'{\i}stica e Inform\'{a}tica - UNICAP}
\email[Oliveira-Lima]{eal@dei.unicap.br}
%================================Oliveira Lima==============================================


\keywords{Combinatorial Algorithms, Complexity, Combinatorial
Optimization, k-Subsets}
%===================================Key Words===============================================


\begin{abstract}
This article presents the \emph{Serial {$k$}-Subset of an
{$n$}-Set Method} (SKSM). The SKSM proposal is to obtain a
{$k$}-subset vector of an {$n$}-set vector from its serial number.
The algorithm that does the SKSM's inverted process is also
showed, getting the serial number from the {$k$}-subset vector.
The article ends with the demonstration of the method exposed
here.
\end{abstract}
%=================================End Abstract=============================================
 \maketitle
%=================================Introduction=============================================
\section*{Introduction}
We call {$\big({n \atop k}\big)$} the number of possibilities to
combine {$n$} things on {$k$} different parts. On analyzed
literature\cite{wi}, we have two ways to do this work on a
sequential form. The first method builds the {$k$}-subsets in a
lexicographic order and the second method obtains the next subset
from its predecessor, subtracting one element of the set and
adding on other element of the subset.

The proposal of this article is presents a method that, put in
action iteratively, it creates a list in a lexicographic order 
given as input the position on the list of possible combinations.
This strategy is more efficient when we desire to get a
combination on a specific position inside the whole list of
combinations.

\section*{Getting the combinations on lexicographic order}

The algorithm \emph{Next {$k$}-subset of an {$n$}-Set}\cite{wi} is
able to create on a simple way the combinations in lexicographic
order in a non-recursive way. The recursive model of this
algorithm will be shown on this section optionally. The current
combination of the recursive model is showed through the
{$showOutput$} method, that is indicating here an output model of
generic data for a combination that is going to have as output
elements that can vary between {$0$} and {$n-1$}. The call to
begin the routine must be done on a {$Combine(0)$} way, taking as
stopped of the recurrence, the moment that the parameter did {$k$}
recurrences. Next, we have a recursive model of the algorithm,
without taking longer on its construction.\\\\
\textbf{Recursive {$k$}-subset of an {$n$}-Set Algorithm}

Algorithm Specifications:
\begin{itemize}

\item {$k$}: Dimension of the subset;

\item {$s$}: Subset vector;

\item {$n$}: Set's cardinality;

\item {$y$}: Auxiliary vector that determines the changes of {$n$}
set elements in subset {$k$}.

\end{itemize}
Routine:\\
\indent {$Combine(i)$} \\
\indent \indent For {$s_i \leftarrow sum(i,i,0)$} to {$n-(k-i)$} do \\
\indent \indent \indent If {$i \neq k-1$} \\
\indent \indent \indent \indent {$Combine(i+1)$} \\
\indent \indent \indent Else \\
\indent \indent \indent \indent {$showOutput$}\\
\indent \indent \indent End If-Else\\
\indent \indent End For\\
\indent \indent {$y_{i+1} \leftarrow 0$}\\
\indent \indent {$y_i \leftarrow y_i + 1$}\\
\indent End Combine. \\\\
\indent Subroutine {$sum(w,j,z)$}\\
\indent \indent For {$i \leftarrow 0$} to {$j$} do \\
\indent \indent \indent {$z \leftarrow z +  y_i$} \\
\indent \indent End For \\
\indent \indent return {$z+w$}.\\
\indent End {$sum$}.

\section*{SKSM Construction}

To build the SKSM, we observe the data output of the subsets' list
created by the \emph{Next {$k$}-subset of an {$n$}-Set} algorithm,
characterizing the repetition's pattern in the elements of subset.
In this case, the pattern can be delineated under a tree model.
The formation law of this tree is showed below.

\subsection*{Definition of the Binomial Tree}

The structure that represents the repetition's pattern is
characterized as a tree formed exclusively
by binomial coefficients.
Let {$\big({n\atop k}\big)_{w}$} be the current node of the tree
with {$w$} label, its descendants are defined by
\begin{equation}
\bigg({n\atop k}\bigg)_{w} \rightarrow \left\{
\begin{array}{l}
\Big({n-1\atop k-1}\Big)_{w+1}\\
\\
\Big({n-2\atop k-1}\Big)_{w+2}\\
\indent \vdots \\
\Big({n-k+1\atop k-1}\Big)_{w + n - k + 1}\\
\end{array} \right.
\end{equation}\\

Each ascendant will have {$n + k -1$} descendants, and the
repetition's pattern of combinations are analyzed through the
insertion of labels in each node of the tree. For each label of
the current node, the descendants nodes enter its labels in
relation to the ascendant node, indicating the value from each
element of the subset founded on the tree.

\subsection*{SKSM Specifications}

The SKSM abstracts the construction of the tree, doing the search
calculating only the binomial coefficients and its relations with
the nodes' labels that were visited on the search. The method does
an external loop attributing to each element of the subset the
result of the subroutine {$element$}. The subroutine {$element$}
does a search on the tree's nodes. The abstraction of the tree is
made through the indices changes of the binomial coefficient
attributed on {$x$} and {$y$}. The method is described below.\\\\
\textbf{Serial {$k$}-Subset of an {$n$}-Set Method}

Algorithm Specifications:
\begin{itemize}

\item {$p$}: {$k$}-dimensional Subset;

\item {$n$}: Cardinality of the Set;

\item {$a$}: Auxiliary variable that is used to check the stopped
of the method;

\item {$x,y$}: Indices of the binomial coefficients functioning as
the formation law of the Binomial Tree;

\item {$r$}: Variable that controls the labels of Binomial Tree;

\item {$s$}: Serial of {$p$} subset;

\item {$C_{n,k}$}: {$\big({n\atop k}\big)$}.

\end{itemize}
Routine:\\
\indent {$x \leftarrow n$} \\
\indent {$y \leftarrow k-1$}\\
\indent {$a \leftarrow r \leftarrow 0$}\\
\indent For {$i \leftarrow 0$} to {$k-1$} do\\
\indent \indent {$p_i \leftarrow element$}\\
\indent End For\\
return {$p$}.\\
End SKSM Routine.\\\\
Subroutine {$element$}\\
\indent For {$j \leftarrow 1 $} to  {$x - y + 1$} do \\
\indent \indent If {$a + C_{x-j,y} < s$}\\
\indent \indent \indent {$a \leftarrow a + C_{x-j,y}$}\\
\indent \indent Else\\
\indent \indent \indent {$x \leftarrow x - j$}\\
\indent \indent \indent {$y \leftarrow y - 1$}\\
\indent \indent \indent {$r \leftarrow r + j$}\\
\indent \indent \indent return {$r$}.\\
\indent \indent End If-Else\\
\indent End For\\
return {$r$}.\\
End Subroutine {$element$}.

\section*{Inverted process of the SKSM}

As many others serial methods from others combinatorial problems
with same importance\cite{me}\cite{me1}\cite{me2}, the SKSM also
has its inverted process. What is giving as input is the vector
that represents the {$k$}-dimensional subset with the cardinality
of the set, called {$n$}, getting as output its position (i.e. the
serial number) on the subsets list in a lexicographic order.


\subsection*{SKSM Inverse Process Specifications}

On the inverted SKSM process, it is taken as the maximum limit of
the internal loop is the difference between the elements of the
subset. This will determine how much the loop will repeat related
to the elements of the subset. The indices of the binomial
coefficient do an appropriate control for the adding among the
nodes be made correctly. The inverted process also abstract the
tree's structure, getting the result by means of local information
related to the subset given as input.\\\\
\textbf{Serial {$k$}-Subset of an {$n$}-Set Method (Inverted
Process)}

Algorithm Specifications:
\begin{itemize}
\item {$p$}: {$k$}-dimensional Subset;

\item {$n$}: Cardinality of the Set;

\item {$x,y$}: Indices of the Binomial Coefficients;

\item {$r$}: Variable that will control the internal loop of the
method;

\item {$s$}: Serial of {$p$} subset;

\item {$C_{n,k}$}: {$\big({n\atop k}\big)$}.

\end{itemize}
Routine:\\
\indent {$x \leftarrow n$}\\
\indent {$y \leftarrow k-1$}\\
\indent {$s \leftarrow 1$}\\
\indent {$r \leftarrow 0$}\\
\indent For {$i \leftarrow 0$} to {$k-1$} do \\
\indent \indent For {$j \leftarrow 1$} to {$p_i - r - 1$} do\\
\indent \indent \indent {$s \leftarrow s + C_{x-j,y}$}\\
\indent \indent End For\\
\indent \indent {$x \leftarrow x - (p_i - r)$}\\
\indent \indent {$y \leftarrow y - 1$}\\
\indent \indent {$r \leftarrow p_i$}\\
\indent End For\\
return s.

\section*{Demonstration}
(This section is not ready)

\begin{thebibliography}{99}
\bibitem{wi} WILF, Herbert S., NIJENHUIS, A.
~Combinatorial Algorithms for computers and calculators. Academic
Press, INC, 1978.
\bibitem{me} MELO, Glaucio G. de M., OLIVEIRA-LIMA, Emerson A. de O. Serial Permutation Method,
(Not published yet),2004.
\bibitem{me1} MELO, Glaucio G. de M., OLIVEIRA-LIMA, Emerson A. de O. Serial Composition Method,
(Not published yet),2004.
\bibitem{me2} MELO, Glaucio G. de M., OLIVEIRA-LIMA, Emerson A. de O. Serial Partition of an n-Set Method,
(Not published yet),2004.
\end{thebibliography}
\end{document}
