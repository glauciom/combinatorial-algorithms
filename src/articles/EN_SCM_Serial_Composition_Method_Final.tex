%=====================================Serial Composition Method==================================
\documentclass{amsart}
\usepackage[latin1]{inputenc} %permite o uso de acentos
\newcommand{\emerson}{Emerson A. de O. Lima}
\newcommand{\glaucio}{Glaucio G. de M. Melo}
\begin{document}

\title[Serial Composition Method]
 {Serial Composition Method}

%==============================================Melo==============================================

\author[Melo]{\glaucio}

\address[Melo]{Departamento de Estat\'{\i}stica e Inform\'{a}tica - UNICAP}

\email[Melo]{glaucio@dei.unicap.br}
%=========================================Oliveira Lima==========================================

\author[Oliveira Lima]{\emerson}

\address[Oliveira Lima]{Departamento de Estat\'{\i}stica e Inform\'{a}tica - UNICAP}

\email[Oliveira Lima]{eal@dei.unicap.br}

%=======================================Key Words===============================================

\keywords{Combinatorial Algorithms, Complexity, Combinatorial Optimization, Composition}

%=======================================Abstract================================================

\begin{abstract}

This article presents the \emph{Serial Composition Method} (SCM). The proposal of the SCM is to obtain the composition vector of an integer {$n$} in {$k$} parts starting from its serial number. It is also represented the method which does the inverse process of the SCM, obtaining the serial number from the composition vector. The article ends with a demonstration of the combinatorial properties which are present in the material here exposed.

\end{abstract}

%========================================Introduction===========================================

\maketitle

\section*{Introduction}
On the combinatorial family, the composition of an integer {$n$} in {$k$} parts is defined by:
\begin{equation}
n = r_1 + r_2 + \ldots + r_k \indent r_i \geq 0 \indent i=1..k
\end{equation}

Where the order of the elements is important on the compositions generations. The \emph{Next Composition Algorithm} \cite{wi} does the task of making composition, obtaining the next composition starting from the one before, interactively working to reach until the last composition of the list. The proposal of the described method on this article is to obtain the vector from the list position, with no need of processing the compositions one by one until getting the expecting vector. 

\section*{Construction of the SCM}

The SCM was built from the repetition patterns which are present on the composition vector throughout its list. It is possible to see that this repetition is done in specific positions on the compositions lists: they can be obtained through calculus. This feature of the composition vector makes possible the cast among the specific positions on the list. This way, it decreases the number of needing interactions to find the vector of a specific position. The calculus of the specific positions is defined below.

\subsection*{Used definitions on the built of SCM}
It is known that the total number of compositions of {$n$} in {$k$} parts is defined by:

\begin{equation} \label{eq1}
J(n,k) = \left(
\begin{array}{ccc}
n + k -1 \\
n\\
\end{array} \right)
\end{equation}
It is also know that:
\begin{equation} \label{eq2}
\sum_{i=0}^{k} \left(
\begin{array}{ccc}
n - i \\
k - i \\
\end{array} \right) =
\left(
\begin{array}{ccc}
n + 1 \\
k\\
\end{array} \right) \;
\end{equation}
Associating \ref{eq1} and \ref{eq2} we have:
\begin{equation} \label{eq3}
\sum_{i=0}^{n} \left(
\begin{array}{ccc}
n + k - 2 - i \\
n - i \\
\end{array} \right) =
\left(
\begin{array}{ccc}
n + k -1 \\
n\\
\end{array} \right) \;
\end{equation}

For a composition vector {$c$} of a length {$k$} from the composition list {$L$}, the modification of {$k$}-tuple element in {$L$} is determined by the index of the partial sum defined on the equation \ref{eq3}. This way, it is possible to know when a component of {$c$} stops to repeat its current element to get modifications. Taking as a basis equation \ref{eq3}, we can see this index modification with the definition of the {$M$} matrix, exposed in \ref{eq4}.

\begin{equation} \label{eq4}
M(n,k) = \left(
\begin{array}{cccc}
%primeira linha
\left(\begin{array}{ccc}n + k -2 \\n\\\end{array} \right) &
\left(\begin{array}{ccc}n + k -3 \\n-1\\\end{array} \right) &
\ldots &
\left(\begin{array}{ccc} k - 2 \\0\\\end{array} \right)\\\\
%Segunda linha
\left(\begin{array}{ccc}n + k -3 \\n\\\end{array} \right) &
\left(\begin{array}{ccc}n + k -4 \\n-1\\\end{array} \right) &
\ldots &
\left(\begin{array}{ccc} k - 3 \\0\\\end{array} \right)\\
\vdots & \vdots & \ddots &\vdots\\
%Segunda linha
\left(\begin{array}{ccc}n\\n\\\end{array} \right) &
\left(\begin{array}{ccc}n -1\\n-1\\\end{array} \right) & \ldots &
\left(\begin{array}{ccc}0\\0\\\end{array} \right)\\
\end{array} \right)
\end{equation} \\
        \begin{itemize}
        \item Each line of {$M$} that corresponds to the elements of the sum on equation \ref{eq3};
        \item From one line to another, the superior element of the binomial is decreased in one unity;
        \item {$M$} possui {$k-1$} lines and {$n+1$} columns.
        \end{itemize}
        
Now let's see bellow how {$M$} is used to represent the SCM execution.

\subsection*{Description of the {$M$} Matrix Raster}

Initially, we have as initial information the serial number of the composition on the list, to obtain the composition vector. Let's suppose {$s$} is the serial related to the input variable of the method to a composition vector with a length {$k$}. The elements of the first line of {$M$} are added until this sum is over {$s-1$}. When it's found a value in the sum which is over {$s-1$}, the corresponding value of the column which the added element immediately before the actual one is found is given to the last element of the composition vector found. After finding the last element (the attributions of each element of the composition vector are done from the end to the beginning until the second element of the composition vector), the next element, that means the element before the last one of the composition vector is found in {$M$} going down in diagonal to the next line of {$M$}, re-starting the counting of indexes starting from the actual column. This is done in a successive way, until the sum of the elements which were visited in {$M$} is {$s-1$}. Let's see an example of this strategy in a matrix {$M(7,5)$} with serial {$s-1 = 282$}, seen in \ref{eq5}. The result of the raster is the composition vector {$c(283)= \{1,0,2,1,3\}$}. It is known that the first element of the vector does not need to be calculated using {$M$}, once it can be obtained from the complement of the sum of the elements already found through the raster done in {$M$}. There is in \ref{eq5} on the first row with the partial sum until the third element (we allocate the number {$3$} on the last position of the list), the second row with only one element of the partial sum (we allocate the number {$1$} on the position before the last one of the list), and so on. The sum of the underlined numbers in \ref{eq5} converged to {$s - 1$}.

\begin{equation} \label{eq5}
M(7,5) = \left(
\begin{array}{cccccccc}
\textbf{\underline{120}} & \textbf{\underline{84}} & \textbf{\underline{56}} & 35 & 20 & 10 & 4 & 1 \\
        36 & 28 & 21 & \textbf{\underline{15}} & 10 & 6 & 3 & 1\\
        8 & 7 & 6 & 5 & \textbf{\underline{4}} & \textbf{\underline{3}} & 2 & 1 \\
        1 & 1 & 1 & 1 & 1 & 1 & 1 & 1 \\
\end{array} \right)
\end{equation}

\section*{Description of the SCM}

Related to implementation, the algorithm proposed here to the SCM is abstract to the structure of {$M$} matrix, defined above. The loop is traced as defined, applying on the first element of the composition vector the corresponding value of the complement of the sum of the elements yet to come. On the subroutine \emph{element} there is the procedure to obtain each element of the composition vector.
\\\\
\textbf{Serial Composition Method (SCM)}
\\
Algorithm specifications:
        \begin{itemize}
        \item {$n$}: Number of the composition (Composition of {$n$} elements in {$k$} parts);
        \item {$k$}: Parts of the composition;
        \item {$s$}: Serial of the composition;
        \item {$a$}: Auxiliary variable which makes the increment to the convergence of the serial number;
        \item {$x,y$}: Auxiliary variables, for the definition of new binomial indexes;
        \item {$C_{i,j}$}: Combination of {$i$} elements {$j$} by {$j$};
        \item {$z$}: Value of the complement, used to apply the value to the first element of the composition vector.
        \end{itemize}
        Routine:
        \\
        \indent {$z \leftarrow a \leftarrow 0$} \\
        \indent {$x \leftarrow n + k - 2$}\\
        \indent {$y \leftarrow n$}\\
        \indent For {$i \leftarrow 0$} to {$k-2$} do\\
        \indent \indent {$c_{k-1-i} \leftarrow element $}\\
        \indent \indent {$z \leftarrow z + c_{k-1-i}$}\\
        \indent End For\\
        \indent {$c_0 \leftarrow n - z$}\\
        return {$c$}.\\
    Subroutine element \\
    \indent For {$j \leftarrow 0$} to {$n-1$} do\\
    \indent \indent If {$a + C_{x-i-j,y-j} \leq s-1$}\\
    \indent \indent \indent {$a \leftarrow a +  C_{x-i-j,y-j}$}\\
    \indent \indent End If\\
    \indent \indent Else\\
    \indent \indent \indent {$x \leftarrow x - j$}\\
    \indent \indent \indent {$y \leftarrow y - j$}\\
    \indent \indent \indent return {$j$}\\
    \indent \indent End Else.\\
  \indent End For \\
    return {$n$}.\\
    End subroutine element.

\section*{Description of the inverse process of SCM}

Related to the inverse process of the SCM, the actual view is the composition serial number obtaining, having as input data the composition vector of a number {$n$} in {$k$} parts. In this case, each component of the composition vector is seen as part of superior intervals of a nested loops, which makes the raster on the composition vector. The inverse process has concise data input related to the data input of the SCM, once the inverse process will not work under the value convergence of a serial number method: everything is already well defined on the composition vector components, known that is only necessary to process the loops with related interactions to each component of the composition vector give as input.
As to the SCM, the proposed algorithm for the inverse process to the SCM works only with indexes referred to the defined matrix on SCM. The use of the indexes abstracting from the matrix structure decreases the memory use and processes only what is needed for the calculus of the trace to be run on {$M$} matrix.
\\\\
\textbf{Serial Composition Method (Inverse process)}
\\
    Algorithm specifications:
    \begin{itemize}
        \item {$n$}: Number of the composition (Composition of {$n$} elements in {$k$} parts);
        \item {$k$}: Parts of the composition;
        \item {$s$}: Serial of the composition;
        \item {$x,y$}: Auxiliary variables, for the definition of new binomial indexes;
        \item {$C_{i,j}$}: Combination of {$i$} elements {$j$} by {$j$}.
        \end{itemize}
        Routine:
        \\
        \indent {$x \leftarrow n + k -2$} \\
        \indent {$y \leftarrow n$} \\
        \indent {$s \leftarrow 1$} \\
        \indent For {$i \leftarrow k-1$} to {$1$} do \\
        \indent \indent For {$ j \leftarrow 0$} to {$c_i -1$} do \\
        \indent \indent \indent {$s \leftarrow s + C_{x-j-[(k-1)-i],y-j}$} \\
        \indent \indent End For \\
        \indent \indent {$x \leftarrow x - c_i$} \\
        \indent \indent {$y \leftarrow y - c_i$} \\
        \indent End For \\
        \indent return s. \\

\section*{Demonstration}
(This section is not ready)

\begin{thebibliography}{99}
\bibitem{wi} WILF, Herbert S., NIJENHUIS, A.
~Combinatorial Algorithms for computers and calculators. Academic
Press, INC, 1978.
\end{thebibliography}
\end{document}

\end{document}
